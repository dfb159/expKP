\subsection{Unsicherheiten}\label{VGuD}

Jegliche Unsicherheiten werden nach GUM bestimmt und berechnet.
Die Gleichungen dazu finden sich in \cref{fig:GUM_combine} und \cref{fig:GUM_formula}.
Für die Unsicherheitsrechnungen wurde die Python Bibliothek \texttt{uncertainties} herangezogen, welche den Richtlinien des GUM folgt.

Für die Unsicherheiten der Parameter in Annäherungskurven wurden die $y$-Unsicherheiten der anzunähernden Werte beachtet und die Methode der kleinsten Quadrate angewandt.
Dafür steht in der Bibliothek die Methode \texttt{scipy.optimize.curve\_fit()} zur Verfügung.

Zur Erstellung von Anpassungskurven wird das Python-Paket \texttt{scipy.odr} verwendet, welches unter anderem die Methoden \texttt{scipy.odr.Model()}, \texttt{scipy.odr.RealData()} und \texttt{scipy.odr.ODR()} zur Verfügung stellt.
Dabei wird auf die sogenannte orthogonale lineare Regression (engl. \emph{Orthogonal Distance Regression} (Abkürzung: ODR)) zurückgegriffen, welche auf der Methode der kleinsten Quadrate basiert und einen modifizierten Levenberg-Marquardt-Algorithmus darstellt.
Für die Parameter von Anpassungskurven und deren Unsicherheiten werden die $x$- und $y$-Unsicherheiten der anzunähernden Werte berücksichtigt und entsprechend gewichtet.
Bei digitalen Messungen wird eine Rechteckverteilung mit $\sigma_X = \frac{\Delta X}{2\sqrt{3}}$ und bei analogem Ablesen eine Dreieckverteilung mit $\sigma_X = \frac{\Delta X}{2\sqrt{6}}$ angenommen.
Die konkreten Werte der jeweiligen Fehlerintervalle $\Delta X$ werden in den entsprechenden Abschnitten angemerkt.

Die jeweiligen $\Delta X$ sind im konkreten Abschnitt zu finden.

\begin{figure}[H]
	\begin{equation*}
		x = \sum_{i=1}^{N} x_i
		;\quad
		\sigma_x = \sqrt{\sum_{i = 1}^{N} \sigma_{x_i}^2}
	\end{equation*}
	\caption{Formel für kombinierte Unsicherheiten des selben Typs nach GUM.}
	\label{fig:GUM_combine}
\end{figure}

\begin{figure}[H]
	\begin{align*}
		f = f(x_1, \dots , x_N)
		;\quad
		\sigma_f = \sqrt{\sum_{i = 1}^{N}\left(\pdv{f}{x_i} \sigma_{x_i}\right) ^2}
	\end{align*}
	\caption{Formel für sich fortpflanzende Unsicherheiten nach GUM.}
	\label{fig:GUM_formula}
\end{figure}

\subsection{Fit-Parameter}
\label{sec:fitval}

\begin{table}[H]
	\centering
	\caption{Fit-Parameter für die Anzahl der Ereignisse $C(\Delta t) = A \cdot \exp{-\frac{(\Delta t - \Delta t_0)^2}{2 \sigma_{\Delta t_0}^2}} + y_0$ bei eingestellter Verzögerung $\Delta t$.}
	\label{fig:fitval1}
	\begin{tabular}{cc|c}
		\toprule
		Amplitude & $A$ & \input{dat/messung1_A.txt} \\
		Peakposition & $\Delta t_0$ & \input{dat/messung1_x0.txt} \\
		Peakbreite & $\sigma_{\Delta t_0}$ & \input{dat/messung1_d.txt} \\
		Offset & $y_0$ & \input{dat/messung1_y0.txt}\\ \bottomrule
	\end{tabular}
\end{table}

\begin{table}[H]
	\centering
	\caption{Fit-Parameter für die Zählrate $C(\theta) = A \cdot \exp{-\frac{(\theta - \theta_0)^2}{2 \sigma_{\theta_0}^2}}$ der Vernichtungsstrahlung.}
	\label{fig:fitval3}
	\begin{tabular}{cc|c}
		\toprule
		Amplitude & $A$ & \input{dat/messung3_A.txt} \\
		Peakposition & $\theta_0$ & \input{dat/messung3_x0.txt} \\
		Peakbreite & $\sigma_{\theta_0}$ & \input{dat/messung3_d.txt} \\ \bottomrule
	\end{tabular}
\end{table}

\begin{table}[H]
	\centering
	\caption{Fit-Parameter für die Funktion $W(\theta) = A \cdot \left( 1 + \frac{1}{8} \cos^2 \theta + \frac{1}{24} \cos^4 \theta \right)$ der theoretischen Winkelkorrelation.}
	\label{fig:fitval4}
	\begin{tabular}{cc|c}
		\toprule
		Amplitude & $A$ & \input{dat/messung4_A.txt} \\ \bottomrule
	\end{tabular}
\end{table}

