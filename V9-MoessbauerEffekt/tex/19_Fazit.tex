\section{Diskussion \& Schlussfolgerung}

	Nun zu der Diskussion der Ergebnisse.
	Dazu zunächst zu den Bauteilen.
	Die aufgenommene Kennlinie der Diode entspricht dem Verhalten der theoretischen Vorhersage, da sich der exponentielle Verlauf gut über einen Fit darstellen lässt. 
	
	\
	
	Der Richtleiter zeigt über den ganzen gemessenen Bereich ein gleichmäßiges Verhalten der Dämpfungen.
	Damit wird der vom Hersteller angegebene Funktionsbereich von \SIrange{0,5}{18}{\giga\hertz} komplett abgedeckt.
	Die Einfügedämpfung liegt nahezu konstant nahe \SI{5}{\decibel}, die Koppeldämpfung ebenso nahe \SI{18}{\decibel} und die Isolationsdämpfung schwankt um \SI{60+-20}{\decibel}.
	Zu ersteren beiden Größen waren von dem Hersteller \SI{10}{\decibel} und \SIrange{10}{15}{\decibel} gegeben.
	Damit weichen die Messergebnisse leicht von den Angaben ab.
	
	\
	
	Für den Isolator war ein Frequenzbereich von \SIrange{3}{8,5}{\giga\hertz} angegeben, in dem die Sperrdämpfung größer als \SI{20}{\decibel} betragen sollte.
	Dies stimmt mit der Messung überein, da diese in einem Bereich von \SIrange{4}{8}{\giga\hertz} konsistent zwischen \SI{60}{\decibel} und \SI{70}{\decibel} liegt.
	Auch die Durchlassdämpfung befindet sich in diesem Bereich, entsprechend den Erwartungen, nahe null.
	Außerhalb des angegebenen Bereichs wurde ebenfalls eine Sperrdämpfung über \SI{20}{\decibel} gemessen.
	Da sie außerhalb der Angaben liegen, ist dies jedoch nicht weiter relevant.
	
	\
	
	Auch bei den Zirkulatoren entsprechen Durchlass- und Sperrdämpfung der Erwartung für die angegebenen Frequenzbereiche von \SIrange{2}{4}{\giga\hertz} und \SIrange{4}{8}{\giga\hertz}.
	Bei beiden ist die Durchlassdämpfung dort nahezu null und die Sperrdämpfung mit mindestens \SI{30}{\decibel} größer als die dafür angegeben \SI{20}{\decibel}\footnote{Genauer waren die "schlechtesten" Werte für die Isolierung bei unterschiedlichen Frequenzbereichen und einer Temperatur von \SI{25}{\celsius} angegeben. Sie alle lagen über \SI{20}{\decibel} und unter \SI{22}{\decibel}.}.
	Das periodische Verhalten in der Sperrdämpfung ist auf Interferenzeffekte zurückzuführen, die durch nicht richtig abgeschlossene Kabel und die folgenden Reflexion entstanden sind.
	
	\
	
	Für beide Kabel konnte eine Ausbreitungsgeschwindigkeit, sowie eine Permittivitätszahl bestimmt werden.
	Konkrete Werte sind in Tab. \ref{tab:kabel} aufgeführt.
	Der große Unterschied von Permittivitätszahlen beider Kabel lässt darauf schließen, dass in den Kabeln unterschiedliche Dielektrika verwendet worden sind.
	Durch die hohe Permittivitätszahl des hellen Kabels, könnte es sich bei dessen Dielektrikum um Teflon handeln.
	Über das Material des dunklen Kabels kann außer der groben Vermutung eines in Schaum eingebrachten Isoliergases keine Aussage gemacht werden (Vergleich mittels \cite{permit1} und \cite{permit2}).
	
	\
	
	Abschließend lässt sich sagen, dass die Ziele dieser Untersuchung erreicht wurden, da die ermittelten Dämpfungen der Bauteile gut mit den Angaben übereinstimmen, teilweise sogar größere Bandbreiten ergeben und auch die Eigenschaften der Koaxialkabel eindeutig bestimmt werden konnten.
	 
