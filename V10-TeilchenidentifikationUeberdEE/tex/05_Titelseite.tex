% Autor: Simon May
% Datum: 2017-10-05

% Befehl, um die E-Mail-Adressen auf der Titelseite darzustellen
\makeatletter
\newcommand*{\protokollemailparse}[1]{%
	\@for\@tempa:=#1\do{%
		\normalsize\email{\@tempa}\\
	}%
}
\makeatother

\begin{titlepage}
	\centering
	{\scshape\LARGE Versuchsbericht zu \par}
	\vspace{1cm}
	{\scshape\huge \varNum {}: \varName\par}
	\vspace{2.5cm}
	{\LARGE \varGruppe\par}
	\vspace{0.5cm}
	{\large \varNameA \,(\email{\varEmailA}) \par}
	{\large \varNameB \,(\email{\varEmailB}) \par}
	{\large \varNameC \,(\email{\varEmailC}) \par}
	\vfill
	durchgeführt am {\large \varDatum}\par
	betreut von {\large \varBetreuer} 
	\vfill	
	{\large \today\par}
\end{titlepage}

% Falls die Datei „res/titelbild.pdf“ existiert, wird sie hier eingefügt
\IfFileExists{res/titelbild.pdf}{
	\publishers{\vspace{2ex}\includegraphics[width=0.75\textwidth]{res/titelbild.pdf}}
}{}

\maketitle
