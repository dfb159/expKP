\section{Schlussfolgerung}

Mit dem $\Delta E$-$E$-Aufbau konnte bestätigt werden, dass es sich bei den abgestrahlten Teilchen des Americium-241 um $\alpha$-Teilchen handelt.
Hierbei ist anzumerken, dass in Abbildung \ref{fig:energieverlust} zwei Messungen nicht auf der theoretischen Kurve für $\alpha$-Teilchen liegen.
Allerdings passt auch keine andere der untersuchten Teilchenarten auf die gemessenen Daten.
Der Vergleich von $\Delta E \beta^2$-Statistiken in Abbildung \ref{fig:debeta} stützt diesen Zusammenhang.

Zusätzlich zum qualitativen Ergebnis des Experimentes wurden die Dicken des Detektors auf \SI{8.52 +- 0.77}{\micro\meter} und die Dicken der drei Folien auf \input{dat/folie_dicke_1.txt}, \input{dat/folie_dicke_2.txt} und \input{dat/folie_dicke_3.txt} bestimmt.
Da die durchlaufende Strecke der $\alpha$-Teilchen durch die Folien mit dem Winkel zunimmt, konnten bei manchen Einstellungen keine Teilchen mehr registriert werden.
In diesem Fall wurde das Teilchen gestoppt und die gesamte Energie in den Mylar-Folien oder dem $\Delta E$-Detektor deponiert.

Die Unsicherheiten folgen vor Allem aus der systematischen Detektorgenauigkeit der Siliziumdetektoren und der statistischen Datenverteilung.
Für eine bessere Systematik kann das aufgenommene Spektrum mit einer weiteren Kalibrationsmessung bei einer einzelnen, wohlbekannten Energie gefaltet werden.
Durch längere Messzeiten können statistische Fehlerquellen verringert werden.
