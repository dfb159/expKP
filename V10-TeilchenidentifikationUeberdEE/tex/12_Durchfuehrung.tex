\section{Versuchsanordnung und -durchführung}

\subsection{Versuchsaufbau}

Der Aufbau dieses Versuchs ist in Abbildung \ref{Aufbau} schematisch dargestellt und besteht aus einer Vakuumkammer, zwei Detektoren, drei Mylar-Folien, einer Strahlungsquelle, der signalverarbeitenden Elektronik sowie einem Computer.
\begin{figure}[H]
	\centering
	\includegraphics[width=1.0\textwidth]{img/Aufbau}
	\caption{In dieser Abbildung ist der schematische Aufbau dieses Versuchs in Form eines Blockschaltbilds dargestellt. Die Abbildung wurde der Versuchsanleitung\cite{wwu} entnommen und anschließend bearbeitet.}
	\label{Aufbau}
\end{figure}
\noindent Die Strahlungsquelle ist Americium-$241$, welches nahezu ausschließlich unter Emission von $\alpha$-Strahlung in angeregte Kernzustände von Neptunium-$237$ zerfällt.
Bei den beiden Detektoren handelt es sich um mit Hochspannung betriebene Halbleiterdetektoren, genauer gesagt Si-Oberflächensperrschicht-Detektoren.
Dabei ist der von einem einfallenden $\alpha$-Teilchen zuerst getroffene Detektor ungefähr $7\,$cm dick und wird als $\Delta E$-Detektor bezeichnet.
Der dem $\Delta E$-Detektor nachstehende $E$-Detektor ist in etwa $20\,$cm dick.

%Innerhalb der Vakuumkammer ist die Strahlungsquelle auf den Folienhalter gerechtet.
%Hinter dem Folienhalter ist der $\Delta E$-Detektor gefolgt vom $E$-Detektor positioniert.
%Die Folien und der $\Delta E$-Detektor sind aus dem Strahlengang entfernbar.
%Durch die Drehung der Folien lässt sich deren effektive Dicke verändern, sodass die Energie der hindurchgehenden $\alpha$-Teilchen variiert wird und kontinuierliche $\alpha$-Energien entstehen.
%Computer mit ADC und CAMAC-Karte.
%Diese ist über einen ADC (engl. \emph{analog-to-digital converter}) mit dem Computer verbunden, welcher über eine entsprechende Datenaufnahme-Software verfügt und somit den Zähler bildet.

\subsection{Elektronik}



\subsection{Durchführung}


