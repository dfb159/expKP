\section{Conclusions}

In this section, previous results are summed up and discussed in more depth.

The mean lifetime measurement gives a theory-matching result up to the second decimal place. Here, measured lifetime is a lot less than the theoretical value for free muons. It is also out-of-range of the calculated standard deviation. As already discussed in theory section, this might be explained with capturing processes of negatively charged muons which lower the total mean lifetime of muons. However, this cannot be verified from the made measurements, further must be taken.

Simply looking at the data for the characteristics of the detected energy-loss-spectra of muons at different zenith angles (table \ref{LandauFitParameters}), one can verify two theoretically predicted statements: The amplitude of the spectra decreases, while the position of the maximum increases for larger angles. 

The amplitudes generally match a $\cos^2$-distribution, too. Nevertheless, four data points are way two few to give statistical evidence. Furthermore, the data point at 75° matches the predicted distribution at least which might indicate that it is not proper for large angles.
	 
