\section{Einleitung}

In der Kern- und Teilchenphysik ist es oftmals notwendig Teilchen zu identifizieren. Der im Folgenden vorgestellte $\Delta E$-$E$-Aufbau bietet die Möglichkeit dies zu tun.

Konkret wird der Energieverlust $\Delta E$ von Teilchen in einem dünnen Detektor gemessen, bevor sie ihre Restenergie $E-\Delta E$ in einem zweiten Detektor deponieren. Der Zusammenhang $\Delta E(E)$ ist charakteristisch für eine Teilchensorte und lässt sich von theoretischer Seite mithilfe der Bethe-Bloch-Formel berechnen. Durch Vergleich von Theorie und Experiment lässt sich so hier beispielhaft verifizieren, dass es sich bei $^{241}$Am tatsächlich um einen Alpha-Strahler handelt. 
%Soll enthalten:

%Zusammenhang:
%Ziel:
%Problem:
%Lösungsansatz: