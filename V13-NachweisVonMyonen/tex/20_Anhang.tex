\appendix
\section{Appendix}
\label{sec:anhang}

\subsection*{Uncertainties}

All uncertainties are calculated using the GUM standards \cite{gum}.
The used equations are given in figures (\ref{eq:GUM_combine}) and (\ref{eq:GUM_formula}).

\begin{figure}[H]
	\centering
	\begin{align}
		x = \sum_{i=1}^{N} x_i
		;\quad
		\sigma_x = \sqrt{\sum_{i = 1}^{N} \sigma_{x_i}^2}
		\label{eq:GUM_combine}
	\end{align}
	\caption*{Equation for combined uncertainties of the same variable according to GUM \cite{gum}.}
	\label{fig:GUM_combine}
\end{figure}

\begin{figure}[H]
	\centering
	\begin{align}
		f = f(x_1, \dots , x_N)
		;\quad
		\sigma_f = \sqrt{\sum_{i = 1}^{N}\left(\pdv{f}{x_i} \sigma_{x_i}\right) ^2}
		\label{eq:GUM_formula}
	\end{align}
	\caption*{Equation for continued uncertainties according to GUM \cite{gum}.}
	\label{fig:GUM_formula}
\end{figure}

\noindent For the calculation of uncertainties, the python library \texttt{uncertainties} is used.
It follows the guidelines of the GUM standards and additionally uses higher order approximations for the continued uncertainty as given in equation(\ref{eq:GUM_formula}).

For digital measurements a rectangular distribution with $\sigma_X = \frac{\Delta X}{2\sqrt{3}}$ is used, for analog measurements a triangular distribution with $\sigma_X = \frac{\Delta X}{2\sqrt{6}}$ is used.
\\For all spectra $\sigma_N = \max\{\sqrt{N}, 1\}$ is used.

\subsection*{Fitting models}

For fitting the data to formulas with multiple free variables, the python library \texttt{scipy.odr} is used.
To connect the data to a given model, the specific libraries \texttt{scipy.odr.Model()}, \texttt{scipy.odr.RealData()} and \texttt{scipy.odr.ODR()} are used.
It will perform an orthogonal distance regression (short: ODR) and a least squares fit with a slightly modified Levenberg-Marquard-algorithm.
ODR uses both $x$ and $y$ uncertainties within the fitting algorithm.
