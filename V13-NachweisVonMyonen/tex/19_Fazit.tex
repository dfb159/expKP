\section{Conclusions}

In this section previous results are summed up and discussed in more depth.

The mean lifetime measurement gives a theory-matching result up to the second decimal place.
Here, the measured lifetime is a lot less than the theoretical value for free muons.
It is also out-of-range of the calculated standard deviation.
As already discussed in theory section, this might be explained by the capturing processes of negatively charged muons which lower the total mean lifetime of muons.
However, this cannot be verified with the measurements made in this experiment.
Further measurement data needs to be taken.

Simply looking at the data for the characteristics of the detected energy loss spectra of muons at different zenith angles (table \ref{LandauFitParameters}), one can verify two theoretically predicted statements: For larger angles the amplitude of the spectra decreases, while the position of the maximum increases. 

The amplitudes generally match a $\cos^2 (\theta)$-distribution, too.
Nevertheless, four data points are far too few to give statistical evidence.
Furthermore, the data point at $\theta = 75\,$° matches the predicted $\cos^2 (\theta)$-distribution least which might indicate that it is not proper for large angles.
	 
