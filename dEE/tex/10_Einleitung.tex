\section{Einleitung}

Analog zum Schalenmodell für die Elektronen in der Atomhülle, lässt sich auch für den Atomkern ein solches definieren.
Auch hier treten gegenüber dem Grundzustand angeregte Zustände auf, die durch verschiedene Quantenzahlen (Hauptquantenzahl, Kernspin, ...) gekennzeichnet sind. Nach $\alpha$- und $\beta$-Zerfällen liegen die die Kerne oft in einem angeregten Zustand vor.
Die "Abregung" geschieht meist über die Aussendung von $\gamma$-Quanten, also hoch-energetischen Photonen.

Durch die Detektion der emittierten Photonen lassen sich Rückschlüsse auf die Charakteristiken der angeregten Zustände ziehen. Konkret wird ein angeregter Kern untersucht der über ein Zwischenniveau und unter Aussendung von zwei Photonen in den Grundzustand zerfällt. Die Winkelverteilung ist charakteristisch für die statische Verteilung der magnetischen Drehimpulsquantenzahlen.

Ziel dieses Versuchs ist es, ein besseres Verständnis von Kernniveaus und ihren Quantenzahlen zu bekommen, sowie den Zusammenhang zu den korrespondierenden Winkelverteilungen der emittierten Photonen einer Zerfallskaskade zu untersuchen.

%Soll enthalten:

%Zusammenhang:
%Ziel:
%Problem:
%Lösungsansatz: