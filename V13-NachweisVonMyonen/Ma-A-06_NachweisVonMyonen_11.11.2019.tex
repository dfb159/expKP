% Autor: Chris Lippe, Jonathan Sigrist, Jannik Tim Zarnitz
% Datum: 2019-11
% basiert auf der Vorlage für Versuchsprotokolle von Simon May

\documentclass[
a4paper,                % Papierformat (DIN A4)
titlepage=firstiscover, % Separate Titelseite
captions=tableheading,  % \caption bei Tabellen immer als Überschrift setzen
toc=bibliography,       % Literaturverzeichnis im Inhaltsverzeichnis aufführen
toc=listof,             % Abbildungsverzeichnis etc. im Inhaltsverzeichnis aufführen
oneside,                % Einseitig
%twoside,               % Zweiseitig
%twocolumn,             % Zweispaltig
automark,               % Abschnittstitel automatisch in Kopfzeile einfügen
12pt,                   % Schriftgröße (beliebige Größen mit „fontsize=Xpt“)
ngerman, english,       % Sprache für z.B. Babel; ausgewählt: (letztgenannt)
%draft=true             % Entwurf-Modus; markiert zu lange und zu kurze Zeilen
parskip = half,         % Abstand nach Absatz
]{scrartcl}

% Autor: Simon May
% Datum: 2017-10-04

% --- Pakete einbinden
% --- Pakete erweitern LaTeX um zusätzliche Funktionen.
%     Dies ist ein Satz nützlicher Pakete.

% Silbentrennung etc.; Sprache wird durch Option bei \documentclass festgelegt
\usepackage{babel}
\usepackage{iftex}
\ifLuaTeX
	% Schriftart (Latin Modern)
	\usepackage{fontspec}
	\fontspec{Latin Modern Roman}
\else
	% Verwendung der Zeichentabelle T1 (für Sonderzeichen etc.)
	\usepackage[T1]{fontenc}
	% Legt die Eingabe-Zeichenkodierung fest, z.B. UTF-8
	\usepackage[utf8]{inputenc}
	% Schriftart (Latin Modern)
	\usepackage{lmodern}
	% Zusätzliche Sonderzeichen
	\usepackage{textcomp}
\fi

\usepackage{upgreek}

% Nutzen von +, -, *, / in \setlength u.ä. (z.B. \setlength{\a + 3cm})
\usepackage{calc}
% Wird benötigt, um \ifthenelse zu benutzen
\usepackage{xifthen}
% Optionen für eigene definierte Befehle
\usepackage{xparse}

% Verbessertes Aussehen des Schriftbilds durch kleine Anpassungen
\usepackage{microtype}
% Automatische Formatierung von Daten
\usepackage[useregional]{datetime2}
% Wird für Kopf- und Fußzeile benötigt
\usepackage{scrlayer-scrpage}
% Einfaches Wechseln zwischen unterschiedlichen Zeilenabständen
\usepackage{setspace}
% Optionen für Listen (enumerate, itemize, …)
\usepackage{enumitem}
% Automatische Anführungszeichen
\usepackage{csquotes}
% Zusätzliche Optionen für Tabellen (tabular)
\usepackage{array}

% Mathepaket (intlimits: Grenzen über/unter Integralzeichen)
\usepackage[intlimits]{amsmath}
% Mathe-Symbole, \mathbb etc.
\usepackage{amssymb}
% Weitere Mathebefehle
\usepackage{mathtools}
% „Schöne“ Brüche im Fließtext
\usepackage{xfrac}
% Ermöglicht die Nutzung von \SI{Zahl}{Einheit} u.a.
\usepackage{siunitx}
% Ermöglicht Nutzung von \pdv als Ableitungen
\usepackage{physics}
% Definition von Unicode-Symbolen; Nach [utf8]inputenc laden!
\usepackage{newunicodechar}
% Unicode-Formeln mit pdfLaTeX
\input{tex/99_pdflatex_unicode-math.tex}

% Farben
\usepackage{xcolor}
% Einbinden von Grafiken (\includegraphics)
\usepackage{graphicx}
% .tex-Dateien mit \includegraphics einbinden
\usepackage{gincltex}
% Größere Freiheiten bei Dateinamen mit \includegraphics
\usepackage{grffile}
% Abbildungen im Fließtext
\usepackage{wrapfig}
% Zitieren, Bibliographie (Biber als Bibliographie-Programm verwenden!)
\usepackage[backend=biber]{biblatex}
% Abbildungen nebeneinander (subfigure, subtable)
\usepackage{subcaption}
\usepackage{float}
\usepackage{booktabs}

% Verlinkt Textstellen im PDF-Dokument (sollte am Ende geladen werden)
\usepackage[unicode]{hyperref}
% „Schlaue“ Referenzen (nach hyperref laden!)
\usepackage{cleveref}
%PDF einbinden
%\usepackage{pdfpages}
%Graphiken zeichnen
%\usepackage{tikz}
%\usetikzlibrary{angles,quotes,babel,3d}

\input{tex/02_EigeneBefehle.tex}
% Autor: Simon May
% Datum: 2016-10-13
% Der Befehl \newcommand kann auch benutzt werden, um „Variablen“ zu definieren:

% Nummer laut Praktikumsheft:
\newcommand*{\varNum}{V7}
% Name laut Praktikumsheft:
\newcommand*{\varName}{$\gamma$-$\gamma$-Winkelkorrelation}
% Datum der Durchführung:
\newcommand*{\varDatum}{18.11.2019}
% Autoren des Protokolls:
\newcommand*{\varAutor}{C. Lippe, J. Sigrist, J. T. Zarnitz}
\newcommand*{\varNameA}{Chris Lippe}
\newcommand*{\varNameB}{Jonathan Sigrist}
\newcommand*{\varNameC}{Jannik Tim Zarnitz}
% Nummer der eigenen Gruppe:
\newcommand*{\varGruppe}{Gruppe Ma-A-06}
% E-Mail-Adressen der Autoren (kommagetrennt ohne Leerzeichen!):
\newcommand{\varEmail}{c\_lipp02@wwu.de,j\_sigrist@wwu.de,j\_zarn02@wwu.de}
\newcommand{\varEmailA}{c\_lipp02@wwu.de}
\newcommand{\varEmailB}{j\_sigrist@wwu.de}
\newcommand{\varEmailC}{j\_zarn02@wwu.de}
%betreuer Name
\newcommand{\varBetreuer}{\normalsize betreut von Benjamin Hetz}
% E-Mail-Adresse anzeigen (true/false):
\newcommand*{\varZeigeEmail}{true}
% Kopfzeile anzeigen (true/false):
\newcommand*{\varZeigeKopfzeile}{true}
% Inhaltsverzeichnis anzeigen (true/false):
\newcommand*{\varZeigeInhaltsverzeichnis}{true}
% Literaturverzeichnis anzeigen (true/false):
\newcommand*{\varZeigeLiteraturverzeichnis}{true}

\newboolean{showEmail}
\setboolean{showEmail}{\varZeigeEmail}
\newboolean{showHeader}
\setboolean{showHeader}{\varZeigeKopfzeile}
\newboolean{showTOC}
\setboolean{showTOC}{\varZeigeInhaltsverzeichnis}
\newboolean{showBibliography}
\setboolean{showBibliography}{\varZeigeLiteraturverzeichnis}


% --- Einstellungen
% -- LaTeX/KOMA
% 1,5-facher Zeilenabstand
\onehalfspacing
\recalctypearea
% Schrift bei Bildunterschriften ändern
\addtokomafont{caption}{\small}
\addtokomafont{captionlabel}{\bfseries}
% Nummerierung der Formeln entsprechend des Abschnitts (z.B. 1.1)
\numberwithin{equation}{section}
% „Verwaiste“ Zeilen am Seitenanfang/-Ende stärker vermeiden
\clubpenalty=1000
\widowpenalty=1000
% Auf mehrere Seiten aufgespaltene Fußnoten stärker vermeiden
\interfootnotelinepenalty=3000

% -- csquotes
% Anführungszeichen automatisch umwandeln
\MakeOuterQuote{"}

% -- siunitx
\sisetup{
	locale=DE,
%	separate-uncertainty,
	output-product=\cdot,
	quotient-mode=fraction,
	per-mode=fraction,
	fraction-function=\sfrac
}

% -- hyperref
\hypersetup{
	% Links/Verweise mit Kasten der Dicke 0.5pt versehen
	pdfborder={0 0 0.5}
	pdftitle={Versuchsprotokoll: \varName},
	pdfauthor={\varAutor},
	pdfsubject={Masterpraktikum},
	pdfkeywords={Physik, Münster, Praktikum, Versuchsprotokoll}
}

% -- cleveref
\crefname{equation}{}{}
\Crefname{equation}{}{}

% -- biblatex (Literaturverzeichnis)
\IfFileExists{res/literatur.bib}{
	\addbibresource{res/literatur.bib}
}{}

% Innenseite der Fußzeile
\ifoot{}
% Mitte der Fußzeile          
\cfoot{-~\pagemark~-}
% Außenseite der Fußzeile
\ofoot{}

% Kopf- und Fußzeile konfigurieren
\ifthenelse{\boolean{showHeader}}{
	\KOMAoptions{headsepline}
	\recalctypearea
	\automark{section}
	% Innenseite der Kopfzeile
	\ihead{}
	% Mitte der Kopfzeile
	\chead{}
	% Außenseite der Kopfzeile
	\ohead{\headmark}
}{}

\renewcommand\maketitle{}
\DeclareSIUnit\year{yr}
\DeclareSIUnit\days{d}
\DeclareSIUnit\decibelmilliwatt{dBm}
\bibliography{res/literatur}
\setlength\parindent{0pt}


\begin{document}
	
	% Römische Seitenzahlen für Titelseite/Inhaltsverzeichnis
	\pagenumbering{roman}
	% Zunächst ohne Kopf-/Fußzeile
	\pagestyle{scrplain}
	
	% --- Titelseite einbinden
	%     Falls die Datei „res/titelbild.pdf“ existiert, wird sie auf der Titelseite
	%     eingefügt
	\IfFileExists{tex/05_Titelseite.tex}{
		% Autor: Simon May
% Datum: 2017-10-05

% Befehl, um die E-Mail-Adressen auf der Titelseite darzustellen
\makeatletter
\newcommand*{\protokollemailparse}[1]{%
	\@for\@tempa:=#1\do{%
		\normalsize\email{\@tempa}\\
	}%
}
\makeatother

\begin{titlepage}
	\centering
	{\scshape\LARGE Experimental report \par}
	\vspace{1cm}
	{\scshape\huge \varNum {} -- \varName\par}
	\vspace{2.5cm}
	{\LARGE \varGruppe\par}
	\vspace{0.5cm}
	{\large \varNameA \,(\varEmailA) \par}
	{\large \varNameB \,(\varEmailB) \par}
	{\large \varNameC \,(\varEmailC) \par}
	\vfill
	measurements taken from {\large \varDatumA} to {\large \varDatumB}\par
	{supervised by \large \varBetreuer} 
	\vfill	
	{\large \today\par}
\end{titlepage}

% Falls die Datei „res/titelbild.pdf“ existiert, wird sie hier eingefügt
\IfFileExists{res/titelbild.pdf}{
	\publishers{\vspace{2ex}\includegraphics[width=0.75\textwidth]{res/titelbild.pdf}}
}{}

\maketitle

	}{}
	
	% --- Inhaltsverzeichnis einbinden
	\ifthenelse{\boolean{showTOC}}{
		\tableofcontents
		\clearpage
	}{}
	
	% Zurücksetzen der Seitenzahlen auf arabische Ziffern
	\pagenumbering{arabic}
	% Ab hier mit Kopf- und Fußzeile
	\pagestyle{scrheadings}
	
	% --- Den Inhalt der Arbeit einbinden
	\section{Introduction}

Earth is permanently bombarded with cosmic rays. Primary cosmic rays consist of protons (90\,\%), $\alpha$-particles (12,5\,\%) and heavier nuclei (2,5\,\%) \cite{wwu}. Through interaction of primary rays with the Earth's atmosphere, secondary rays are produced. Further nuclear processes, which will be discussed in detail later on, lead to the production of muons (amongst others). These muons are supposed to be detected within the following experiments.

In particular, the mean lifetime of muons stopped in the used detector module is measured, as well as the dependency of cosmic rays' spectrum and count rate on zenith angle. This way, not only the existence of cosmic rays and correctness of nuclear processes to produce muons is proven, but also the characteristics of cosmic rays detected at sea level can be quantified.


%Soll enthalten:

%Zusammenhang
%Ziel
%Problem
%Lösungsansatz
	\newpage
	\section{Theorie}
	
Im Folgenden sollen zunächst die theoretischen Grundlagen für die nachfolgenden experimentellen Untersuchungen erörtert werden.
Die vorgestellte Theorie basiert auf der ausgehändigten Versuchsanleitung \cite{wwu}.

\subsection{Fermigasmodell des Atomkerns}
Eine Möglichkeit, einen Atomkern modellhaft zu beschreiben, bildet das Fermigasmodell, benannt nach Enrico Fermi. Der Kern wird dabei als freies Nukleonengas beschrieben. Die Nukleonen wechselwirken untereinander nicht, sondern befinden sich in zwei Potentialtöpfen, einer für die Protonen und einer für die Neutronen. Im Gegensatz zum Vielelektronenproblem in der Atomphysik, handelt es sich bei Neutronen und Protonen um unterscheidbare Teilchenarten, weshalb sie in zwei unterschiedlichen Potentialtöpfen sitzen. In den Potentialtöpfen werden die möglichen Zustände bis zur Fermi-Energie $E_F$ aufgefüllt. Für symmetrische Kerne ist diese:
\begin{align*}
E_F\approx 33\,\textnormal{MeV}
\end{align*}
\noindent Da es sich bei Protonen und Neutronen um Fermionen handelt, können die verschiedenen Energieniveaus nur von jeweils zwei Teilchen mit gegensätzlichem Spin besetzt werden. Außerdem ist der Potentialtopf der Protonen nicht so tief wie der der Neutronen, da die Protonen zusätzlich noch der Coulombabstoßung unterliegen. In Abbildung \ref{fermigas} ist das Potentialschema des Fermigasmodels nocheinmal graphisch dargestellt.

\begin{figure}[h]
	\centering
	\includegraphics[width=1.0\textwidth]{img/fermigas}
	\caption{Graphische Darstellung der beiden Potentialtöpfe für Neutronen und Protonen im Fermigasmodell. \cite{fermi}}
	\label{fermigas}
\end{figure}

\subsection{Alpha-Zerfall}

Der Alpha$\left( \alpha\right) $-Zerfall wurde erstmals von Ernest Rutherford beobachtet und stellt die Emission eines Heliumkerns dar.Er tritt nur bei relativ schweren Kernen auf. Dies kann man durch die hohe Bindungsenergie von ca. 7\,MeV des Heliumkerns begründen. Mit steigender Massenzahl $A$ nimmt die Bindungsenergie ab, weshalb sich für schwere Kerne im Inneren zwei Protonen und zwei Neutronen als ein Heliumkern \glqq abkapseln\grqq\ können. Dieser Heliumkern ist jedoch nicht frei, sondern im Kernpotential gebunden. 

\noindent Stellt man sich das Potential wie in Abbildung \ref{fermigas} nach dem Fermigasmodell vor, so befindet sich das $\alpha$-Teilchen noch im Rechteckpotential des Kerns, jedoch oberhalb der 0\,MeV-Linie. Daher lässt sich quantenmechanisch eine Wahrscheinlichkeit ungleich null berechnen, mit der das $\alpha$-Teilchen auf die rechte Seite des Coulombpotentials durchtunnelt. Diese Wahrscheinlichkeit bestimmt die Zerfallsdauer und lässt durch den sogenannten Gamow-Faktor näherungsweise berechnen. Das Grundprinzip des Alpha-Zerfalls wird also mit dem Fermigasmodell verständlich. Als allgemeine Zerfallsgleichung lässt er sich schreiben als:
\begin{align*}
^A_ZX_N\longrightarrow \,^{A-4}_{Z-2}Y_{N-2} \,+ \,^4_2\textnormal{He}_2
\end{align*}
Es handelt sich also um einen Zweikörperzerfall. Aus Energie- und Impulserhaltung folgt, dass die Energie der Alpha-Teilchen diskret sein muss. Es kann jedoch mehrere diskrete Energielinien geben, je nachdem ob der Tochterkern nach dem Zerfall in einem angeregten Zustand vorliegt oder nicht. 

Der Gamow-Faktor für ein Teilchen der Masse $m$ und Energie $E$ beim Tunneln durch ein Potential $V(x)$ zwischen den Punkten $a$ und $b$ ist gegeben durch:
\begin{align}
	T=\exp\left[ -\frac{2}{\hbar}\int_{a}^{b}\sqrt{2m\left( V(x)-E\right) }\,\text{d}x\right] 
\end{align}

\subsection{Wechselwirkung von Alpha-Strahlung mit Materie}

Schwere geladene Teilchen (d.\,h. keine Elektronen/Positronen) verlieren beim Durchqueren eines Festkörpers die meiste Energie durch inelastische Kollisionen mit den Elektronen des Festkörpers. Bei sehr schweren Kernbruchstücken ist zudem die Wechselwirkung mit den Kernen des Mediums nicht zu vernachlässigen. Die Kollisionen mit den Elektronen können zudem in weiche und harte unterteilt werden. Bei weichen Kollisionen kommt es nur zur einer Anregung, bei harten gar zu einer Ionisation. 

\subsubsection{Bethe-Bloch-Formel}

Der Energieverlust schwerer geladener Teilchen kann durch die Bethe-Bloch-Formel beschrieben werden. Sie lautet:
\begin{align}
	\label{eq:bethebloch}
	-\derivative{E}{x}=K\rho \frac{Z}{A}\frac{z^2}{\beta^2}\left[ \ln(\frac{2m_e\gamma^2v^2W_\text{max}}{I^2})-2\beta^2-\delta -2\frac{C}{Z}\right] 
\end{align}
Dabei ist $K=2\pi N_A r_e^2m_ec^2=0,1535\,$MeV\,cm$^2$\,g$^{-1}$. Eine Erklärung aller auftauchenden Größen findet sich in der nachfolgenden Tabelle \ref{tab:bethebloch}. Außerdem zeigt Abbildung \ref{bethebloch} einen beispielhaften Kurvenverlauf der Bethe-Bloch-Formel für Myonen in Kupfer.

\begin{figure}[h]
	\centering
	\includegraphics[width=\textwidth]{img/BetheBloch}
	\caption{Beispielhafter Kurvenverlauf der Bethe-Bloch-Formel für Myonen in Kupfer. \cite{bethebloch}}
	\label{bethebloch}
\end{figure}

Um die einzelnen Terme besser zu verstehen, wird eine sehr verkürzte Herleitung durch klassische Überlegungen vorgestellt: Ein Teichen der Ladung $z\cdot e$, Masse $m$ und Geschwindigkeit $v$ fliege durch ein Medium und dabei an einem atomaren Elektron im Abstand $b$ vorbei. Dieses Elektron sei ferner frei und anfangs in Ruhe. Das eintreffende Teilchen werde außerdem aufgrund seiner viel größeren Masse ($M\gg m_e$) nicht aus seiner Bahn gelenkt. Damit kann man nun den Impulsübertrag an das Elektron berechnen:
\begin{align*}
	I=\int F \,\text{d}t=e\int E_\perp \,\text{d}t=e\int E_\perp \derivative{t}{x} \,\text{d}x=e\int E_\perp \frac{1}{v} \,\text{d}x=\frac{2ze^2}{bv}
\end{align*}
Das letzte Integral wurde mit dem Gauß'schen Integralsatz gelöst. Die aufgenommene Energie des Elektrons ist damit gegeben durch:
\begin{align*}
	\Delta E(b)=\frac{I^2}{2m_e}=\frac{2z^2e^4}{m_e v^2b^2}
\end{align*}
Man führt nun die Elektronendichte $N_e$ ein. Dann ist der infinitesimale Energieverlust eines Teilchens an Elektronen zwischen $b$ und $b+\text{d}b$ gegeben durch:
\begin{align*}
	-\text{d}E(b)=\Delta E(b)N_e \,\text{d}V= \frac{4\pi z^2e^4}{m_e v^2b}N_e \,\text{d}b \,\text{d}x
\end{align*}
Die Integration über $b$ lässt sich direkt ausführen:
\begin{align*}
	-\derivative{E}{x}=\frac{4\pi z^2e^4}{m_e v^2}N_e\ln(\frac{b_\text{max}}{b_\text{min}})
\end{align*}
Die Werte von $b_\text{max}$ und $b_\text{min}$ lassen sich physikalisch begründen zu:
\begin{align*}
	b_\text{min}= \frac{ze^2}{\gamma m_e v^2}\qquad\text{und}\qquad b_\text{max}=\frac{\gamma v}{\bar{\nu}}
\end{align*}
Dabei ist $\bar{\nu}$ die mittlere Frequenz aller gebundenen Zustände. Daraus ergibt sich die klassische Energieverlustformel nach Bohr. Die Bethe-Bloch-Formel folgt mithilfe einiger quantenmechanischen Korrekturen, welche hier nicht im Detail besprochen werden sollen.

\begin{table}[h]
	\centering
	\caption{Auftauchende Größen in der Bethe-Bloch-Formel}
	\begin{tabular}{cc}
		$r_e$: & klassischer Elektronenradius \\
		$m_e$: & Elektronenmasse \\
		$N_A$: & Avogadrokonstante \\
		$I$:  & mittleres Anregungspotential \\
		$Z$:  & Kernladungszahl Absorbermaterial \\
		$A$:  & Massenzahl Absorbermaterial \\
		$\rho$: & Dichte Absorbermaterial \\
		$z$:  & Ladungszahl des eintreffenden Teilchens \\
		$\beta$: & $v/c$ des eintreffenden Teilchens \\
		$\gamma$: & $1/\sqrt{1-\beta^2}$ \\
		$\delta$: & Dichte-Korrekturfaktor \\
		$C$:  & Schalen-Korrekturfaktor \\
		$W_\text{max}$: & max. Energietransfer pro einzelner Kollision \\
	\end{tabular}%
	\label{tab:bethebloch}%
\end{table}%

Für die Auswertung der Bethe-Bloch-Formel sind zusätzlich folgende Zusammenhänge notwendig:
\begin{align}
	W_\text{max}\approx 2m_ev^2\gamma^2\qquad\text{für }M\gg m_e
\end{align}
Der Dichte-Korrekturfaktor lautet:
\begin{align}
	\delta=\begin{cases}
	0 &\qquad\text{für }X<X_0\\
	4,6052\cdot X + C_0 + a(X_1-X)^m &\qquad\text{für }X_0<X<X_1\\
	4,6052\cdot X + C_0  &\qquad\text{für }X>X_1
	\end{cases}
\end{align}
Hierbei ist $X=\log_{10}(\beta \gamma)$. Für Silizium sind $I=173\,$eV, $C_0=-4,44$, $a=0,1492$, $m=3,25$, $X_1=2,87$ und $X_0=0,2014$ \cite{bethebloch}. Der  Schalen-Korrekturfaktor lautet:
\begin{align}
	C(I,\eta=\beta\gamma)=(0,422377\,\eta^{-2}+0,0304043\,\eta^{-4}-0,00038106\,\eta^{-6})\cdot 10^{-6}\,I^2 \nonumber\\
	+(3,850190\,\eta^{-2}-0,1667989\,\eta^{-4}+0,00157955\,\eta^{-6})\cdot 10^{-9}\,I^3
\end{align}
Ferner gilt
\begin{align}
	\gamma = 1 + \frac{E_\text{kin}}{mc^2}\qquad\text{und}\qquad\beta=\sqrt{1-\frac{1}{\gamma^2}}.
\end{align}

\subsubsection{Bragg-Peak}

Aus dieser Abhängigkeit des Energieverlustes folgt die sogenannte Bragg-Kurve. Diese stellt den Energieverlust pro zurückgelegtem Weg dar. Wie in Abbildung \ref{bragg} am Beispiel von Alpha-Teilchen in Luft zu sehen, ist der Energieverlust zunächst gering, da die Teilchen noch relativ viel Energie besitzen. Mit abnehmender Energie steigt der Energieverlust weiter an und die Teilchen verlieren die meiste Energie ganz zum Schluss am Bragg-Peak.

\begin{figure}[H]
	\centering
	\includegraphics[width=0.8\textwidth]{img/bragg}
	\caption{Bragg-Peak von Alpha-Teilchen in Luft. \cite{bragg}}
	\label{bragg}
\end{figure}

\subsection{Halbleiter-Detektoren}

Ein Halbleiter-Detektor funktioniert ähnlich zu einer Gas-Ionisationskammer. Im Gegensatz dazu werden jedoch nicht die Ionisationen eines Gases gemessen, sondern die erzeugten Elektron-Loch-Paare in der Sperrschicht einer Halbleiter-Diode. Die zum Hervorrufen einer Reaktion notwendige Energie ist bei einem Halbleiter-Detektor um eine Größenordnung kleiner, was zu mehr freien Ladungen und einer kleineren statistischen Schwankung führt. Zum besseren Verständnis von Halbleiter-Detektoren werden im Folgenden einige Grundlagen zu Halbleitern und pn-Übergängen vorgestellt.

\subsubsection{Halbleiter}

Im Gegensatz zu Leitern zeichnen sich Halbleiter dadurch aus, dass sie eine Bandlücke zwischen Valenz- und Leitungsband besitzen.
Diese Bandlücke ist bei Raumtemperatur ($T=300\,$K) gerade klein genug ($\sim eV$, abhängig vom Halbleitermaterial und der Temperatur), um Elektronen durch thermische Anregung des Materials vom Valenz- ins Leitungsband zu heben.
Im Vergleich zu Halbleitern ist die Bandlücke bei Isolatoren wiederum um ein Vielfaches größer.
In den folgenden Experimenten wird mit Silizium ein elementarer Halbleiter der vierten Hauptgruppe des Periodensystems betrachtet.
Andere Halbleiter wie Galliumarsenid sind beispielsweise aus Elementen der dritten und fünften Hauptgruppe zusammengesetzt.

Bei Raumtemperatur ist der spezifische Widerstand eines Halbleiters groß, da sich nach dem Bändermodell nur wenige Ladungsträger im Leitungsband befinden.
Durch das Einbringen von Elementen der fünften Hauptgruppe (Donatoren) in das (Halbleiter-)Kristallgitter kommt eine sogenannte n-Dotierung zustande.
Dabei besitzt das eingebrachte Element ein negativ geladenes Elektron zu viel in der äußeren Schale, um sich optimal in das Kristallgitter einfügen zu können.
Dies führt dazu, dass dieses Elektron nicht fest an den Atomrumpf gebunden ist und durch thermische Anregung leicht ins Leitungsband befördert werden kann.
Somit sinkt der spezifische Widerstand und die Leitfähigkeit steigt.
Ebenso kann mit Elementen der dritten Hauptgruppe (Akzeptoren) eine sogenannte p-Dotierung erreicht werden.
Dadurch kommt es zu fehlenden Elektronen im (Halbleiter-)Kristallgitter, sodass mit einer geringen thermischen Anregung ein Elektron von einem benachbarten Atom an diese Fehlstelle springen kann.
Nun liegt die Fehlstelle aber am Nachbar-Atom vor, zu der wiederum ein Elektron eines anderen Atoms springen kann.
Es handelt sich also um ein \glqq bewegliches Loch\grqq , was einem Ladungstransport (\glqq Löcherbewegung\grqq ) gleichkommt.
Daher sinkt der spezifische Widerstand und die Leitfähigkeit steigt.


\subsubsection{Der pn-Übergang}

Für den Übergang zum Halbleiter-Detektor ist der sogenannte pn-Übergang entscheidend. Bei Verwendung als elektrisches Bauteil wird er dabei auch als Diode bezeichnet. Kommen ein p- und ein n-dotierter Halbleiter in Kontakt sorgen die unterschiedlichen Konzentrationen von Elektronen und Löchern dafür, dass die Majoritätsladungsträger in die jeweils anders dotierte Halbleiterschicht diffundieren und dort rekombinieren. Die Atomrümpfe bleiben hingegen an ihren Positionen, wodurch sich im n-dotierten Bereich eine positive Raumladung und im p-dotierten Bereich eine negative Raumladung ergibt. Dieses Potentialgefälle erzeugt im Inneren der Diode eine sogenannte Raumladungszone, welche einer weiteren Diffusion von Majoritätsladungstägern offensichtlich entgegen wirkt.

Im thermodynamischen Gleichgewicht halten sich diese beiden Prozesse die Waage. Das heißt, die Summe des Diffusionsstroms und des entgegengesetzten, durch die Raumladungszone erzeugten Driftstroms ist gleich null. Dies führt auf eine lineare Differentialgleichung erster Ordnung, deren Lösung die bekannte Kennlinie einer Diode ist:
\begin{align}
I=I_0\cdot\left[ \exp\left( \frac{e\,U}{n\,k_\textnormal{B}\,T}\right)  -1\right] 
\end{align} 

\noindent Dabei ist $k_\textnormal{B}$  die Boltzmannkonstante,
$T$ die Temperatur in K, 
$I_0$ der Sättigungsstrom, abhängig von Materialparametern und der Temperatur, sowie
$n$ der Idealitätsfaktor mit $ 1 \leq n < 2$. Er beschreibt die Ausdehnung der Raumladungszone. Bei kleiner Ausdehnung ist $n = 1$ eine gute Näherung.

Liegt der n-Bereich auf positivem Potential $(U<0)$ wird die Diffusionsspannung $-U_\textnormal{Diff}$ verstärkt: Man gerät in den Sperrbereich. Bei sehr hoher Sperrspannungen wird der Stromfluss wird alleine durch den Driftstrom der Minoritätsladungen bestimmt $(I\approx I_0)$. Bei Polung der Diode in Flussrichtung $(U > 0)$ wird die Diffusionsspannung abgebaut und der Stromfluss wächst exponentiell. Bei einer realen Diode müssen außerdem noch Korrekturen durch auftretende Leistungsverluste vorgenommen werden. Diese werden beschrieben durch einen Parallel-/Shuntwiderstand $R_{\textnormal{sh}}$ und einen Serienwiderstand $R_{\textnormal{s}}$:
\begin{align}
I=I_0\cdot\left[ \exp\left( \frac{e\,(U-I\,R_\textnormal{s})}{n\,k_\textnormal{B}\,T}\right)  -1\right] +\frac{(U-I\,R_\textnormal{s})}{R_\textnormal{sh}}
\end{align} 


\subsection{Americium-241 als Quelle}

Die im Experiment verwendete Quelle ist $^{241}$Am. Aufgrund der Lebensdauer
von 432,2 Jahren kann es nicht mehr aus der Natur gewonnen werden, sondern muss synthetisiert werden. Durch Emission eines $\alpha$-Teilchens zerfällt es in $^{237}$Np (Neptunium). Außerdem besteht die Möglichkeit einer spontanen Kernspaltung, jedoch ist die Wahrscheinlichkeit so gering, dass sie hier vernachlässigt werden kann. 

Nach dem Zerfall liegt das entstehende $^{237}$Np nur selten im Grundzustand vor. Mit einer Wahrscheinlichkeit von ca. 84,45(10)\,\% zerfällt das Americium in den zweiten angeregten Zustand von Neptunium. Die aus der Energie- und Impulserhaltung bei einem Zweikörperzerfall resultierende Energie für die Alpha-Teilchen ist damit:
\begin{align*}
	E_\alpha = 5485,56(12)\,\text{keV}
\end{align*}
	\newpage
	\section{Versuchsaufbau und -durchführung} \label{sec:aufbau}

	In diesem Abschnitt werden der Aufbau und die Durchführung des Versuchs behandelt.

\subsection*{Aufbau}
	
%	\begin{figure}[h]
%		\centering
%		\includegraphics[width=.8\textwidth]{img/aufbau.png}
%		\caption{Skizze des Versuchsaufbau.\cite{wwu}}
%		\label{fig:aufbau}
%	\end{figure}

	\begin{figure}[H]
		\centering
		\begin{subfigure}[c]{\textwidth}		
			\centering	
			\includegraphics[width=0.7\textwidth]{img/aufbau.png}
		\end{subfigure}
		
		\begin{subfigure}[c]{\textwidth}
			\centering
			\includegraphics[width=0.7\textwidth]{img/aufbau2.png}
		\end{subfigure}
		
		\caption{Skizze des Versuchaufbaus. 
			Oben zur Messung der einzelnen Komponenten. 
			Unten zur Untersuchung der Eigenschaften von Koaxialkabeln. \cite{wwu}}
		\label{fig:aufbau}
	\end{figure}

	Der Versuch wird wie in \cref{fig:aufbau} dargestellt aufgebaut.
	Zu erkennen sind ein Mikrowellengenerator, das zu vermessende Objekt, die Messdiode und ein Multimeter.
	Bei der Vermessung einzelner Bauteile werden diese direkt zwischen Diode und Mikrowellengenerator geschaltet.
	Um hingegen die Koaxialkabel zu untersuchen, muss das Signal zunächst über einen Zirkulator umgelenkt werden.
	Sowohl Mikrowellengenerator als auch Multimeter sind mit einem Computer verbunden, auf dem die LabVIEW-Programme "Messungen Frequenz" und "Messungen Leistung" zur Aufnahme der Daten vorliegen.
	Verbunden werden Mikrowellengenerator und Multimeter über Koaxialkabel mit Messobjekt und Diode.
	Die Verbindung zu dem Computer erfolgt über GPIB-Kabel.
	Alle nicht verwendeten Anschlüsse werden mit einem angepassten Abschlusswiderstand von \SI{50}{\ohm} versehen, um Reflexionen und somit ungewollte Interferenzerscheinungen zu verhindern.
	
	Für die Messung werden das Digitalmultimeter TTi 1906 und der Mikrowellengenerator Wiltron 6759B verwendet.
	Bei der Diode handelt es sich um eine Schottky-Diode.
	
\subsection*{Durchführung}

	Angefangen wird mit der Messung der Kennlinie der Diode.
	Da sie Bestandteil aller Messungen ist, dient sie als Referenz.
	Hierzu wird kein Messobjekt angeschlossen und das LabVIEW-Programm "Messungen Leistung" verwendet.
	Gemessen wird die Spannung am Multimeter in Abhängigkeit von der Eingangsleistung am Mikrowellengenerator.
	Dabei wird ein Leistungsbereich von \SIrange{-20}{15}{\decibelmilliwatt} gewählt, eine Messpunktzahl von 71 und eine Frequenz von \SI{5}{\giga\hertz} für die elektromagnetische Welle.
	
	\
	
	Die weiteren Messungen erfolgen nach einem ähnlichem Schema.
	Hier wird nun die Leistung bei \SI{10}{\decibelmilliwatt} festgehalten und die Frequenz über das LabVIEW-Programm "Messungen Frequenz" variiert und weiterhin die Spannung an dem Multimeter gemessen.
	Die Frequenzbereiche werden anhand der vorliegenden Bauteilinformationen so gewählt, dass sie den gesamten Bereich abdecken für den sie vorgesehen sind.
	
	Zunächst wird der Richtleiter vermessen.
	Dazu werden über die Torpaare (1-2), (1-4) und (2,4) gemessen, sodass daraus im nächsten Abschnitt die unterschiedlichen Kenngrößen ermittelt werden können.
	
	Als nächstes werden zwei Zirkulatoren vermessen, jeweils einmal in Durchlass- (1-2) und Sperrrichtung (2-1).
	Da es sich um symmetrische Bauteile handelt, wird nicht über gleichwertige Torpaare wie (2-3) gemessen.
	Genau gleich (abgesehen von den Bauteilparametern) verläuft die Messung des Isolators.
	
	Zuletzt wird einer der beiden Zirkulatoren erneut in Durchlassrichtung angeschlossen.
	An dieser Stelle befindet sich jedoch nicht die Diode an Tor 2 sondern ein kapazitives Koppelstück, welches mit einem Koaxialkabel verbunden ist.
	Das Ende des Kabels wird mit einem Abschlusswiderstand kurzgeschlossen.
	Tor 3 wird nun mit der Diode verbunden, von der das Signal wie zuvor zum Multimeter führt.
	Durchgeführt wird die Messung für zwei Koaxialkabel unterschiedlicher Länge wie zuvor, jedoch bei einer festen Leistung von \SI{5}{\decibelmilliwatt}.
	
	\newpage
	\section{Datenanalyse}

	Dieser Abschnitt umfasst die Auswertung der aufgenommenen Daten.

\subsection{Diodenkennlinie}
	
	Lorem ipsum dolor sit amet, consetetur sadipscing elitr, sed diam nonumy eirmod tempor invidunt ut labore et dolore magna aliquyam erat, sed diam voluptua. At vero eos et accusam et justo duo dolores et ea rebum. Stet clita kasd gubergren, no sea takimata sanctus est Lorem ipsum dolor sit amet. Lorem ipsum dolor sit amet, consetetur sadipscing elitr, sed diam nonumy eirmod tempor invidunt ut labore et dolore magna aliquyam erat, sed diam voluptua. At vero eos et accusam et justo duo dolores et ea rebum. Stet clita kasd gubergren, no sea takimata sanctus est Lorem ipsum dolor sit amet. Lorem ipsum dolor sit amet, consetetur sadipscing elitr, sed diam nonumy eirmod tempor invidunt ut labore et dolore magna aliquyam erat, sed diam voluptua. At vero eos et accusam et justo duo dolores et ea rebum. Stet clita kasd gubergren, no sea takimata sanctus est Lorem ipsum dolor sit amet. 
	
	\begin{equation}
		U_k(L)=a W_k\left(B\cdot 10^{\frac{L}{\SI{10}{\decibelmilliwatt}}}\right),
	\end{equation}
	
	Duis autem vel eum iriure dolor in hendrerit in vulputate velit esse molestie consequat, vel illum dolore eu feugiat nulla facilisis at vero eros et accumsan et iusto odio dignissim qui blandit praesent luptatum zzril delenit augue duis dolore te feugait nulla facilisi. Lorem ipsum dolor sit amet, consectetuer adipiscing elit, sed diam nonummy nibh euismod tincidunt ut laoreet dolore magna aliquam erat volutpat. 

\subsection{Bauteile der Hochfrequenztechnik}

	Lorem ipsum dolor sit amet, consetetur sadipscing elitr, sed diam nonumy eirmod tempor invidunt ut labore et dolore magna aliquyam erat, sed diam voluptua. At vero eos et accusam et justo duo dolores et ea rebum. Stet clita kasd gubergren, no sea takimata sanctus est Lorem ipsum dolor sit amet. Lorem ipsum dolor sit amet, consetetur sadipscing elitr, sed diam nonumy eirmod tempor invidunt ut labore et dolore magna aliquyam erat, sed diam voluptua. At vero eos et accusam et justo duo dolores et ea rebum. Stet clita kasd gubergren, no sea takimata sanctus est Lorem ipsum dolor sit amet. Lorem ipsum dolor sit amet, consetetur sadipscing elitr, sed diam nonumy eirmod tempor invidunt ut labore et dolore magna aliquyam erat, sed diam voluptua. At vero eos et accusam et justo duo dolores et ea rebum. Stet clita kasd gubergren, no sea takimata sanctus est Lorem ipsum dolor sit amet. 
	
	Duis autem vel eum iriure dolor in hendrerit in vulputate velit esse molestie consequat, vel illum dolore eu feugiat nulla facilisis at vero eros et accumsan et iusto odio dignissim qui blandit praesent luptatum zzril delenit augue duis dolore te feugait nulla facilisi. Lorem ipsum dolor sit amet, consectetuer adipiscing elit, sed diam nonummy nibh euismod tincidunt ut laoreet dolore magna aliquam erat volutpat. 
	\begin{table}[ht]
		\centering
		\caption{Energieaufspaltungen aller sechs Peaks von Absorber C.} 
		\label{tab:isoC}
		\begin{tabular}{c|ccc}
			\toprule
			       & $v_\text{res}$ & $\Delta v_\text{iso}$ & $\Delta E_\text{iso}$ \\ \midrule
			Peak 1 &      123       &          123          &          123          \\
			Peak 2 &      123       &          123          &          123          \\
			Peak 3 &      123       &          123          &          123          \\
			Peak 4 &      123       &          123          &          123          \\
			Peak 5 &      123       &          123          &          123          \\
			Peak 6 &      123       &          123          &          123          \\ \bottomrule
		\end{tabular}
	\end{table} 
	
	
	%\input{tex/name.tex}
	% --- Fazit des gesammten Versuchs einbinden, falls nötig
	%	\IfFileExists{tex/19_Fazit.tex}{
	%		\section{Diskussion \& Schlussfolgerung}

	Nun zu der Diskussion der Ergebnisse.
	Dazu zunächst zu den Bauteilen.
	Die aufgenommene Kennlinie der Diode entspricht dem Verhalten der theoretischen Vorhersage, da sich der exponentielle Verlauf gut über einen Fit darstellen lässt. 
	
	\
	
	Der Richtleiter zeigt über den ganzen gemessenen Bereich ein gleichmäßiges Verhalten der Dämpfungen.
	Damit wird der vom Hersteller angegebene Funktionsbereich von \SIrange{0,5}{18}{\giga\hertz} komplett abgedeckt.
	Die Einfügedämpfung liegt nahezu konstant nahe \SI{5}{\decibel}, die Koppeldämpfung ebenso nahe \SI{18}{\decibel} und die Isolationsdämpfung schwankt um \SI{60+-20}{\decibel}.
	Zu ersteren beiden Größen waren von dem Hersteller \SI{10}{\decibel} und \SIrange{10}{15}{\decibel} gegeben.
	Damit weichen die Messergebnisse leicht von den Angaben ab.
	
	\
	
	Für den Isolator war ein Frequenzbereich von \SIrange{3}{8,5}{\giga\hertz} angegeben, in dem die Sperrdämpfung größer als \SI{20}{\decibel} betragen sollte.
	Dies stimmt mit der Messung überein, da diese in einem Bereich von \SIrange{4}{8}{\giga\hertz} konsistent zwischen \SI{60}{\decibel} und \SI{70}{\decibel} liegt.
	Auch die Durchlassdämpfung befindet sich in diesem Bereich, entsprechend den Erwartungen, nahe null.
	Außerhalb des angegebenen Bereichs wurde ebenfalls eine Sperrdämpfung über \SI{20}{\decibel} gemessen.
	Da sie außerhalb der Angaben liegen, ist dies jedoch nicht weiter relevant.
	
	\
	
	Auch bei den Zirkulatoren entsprechen Durchlass- und Sperrdämpfung der Erwartung für die angegebenen Frequenzbereiche von \SIrange{2}{4}{\giga\hertz} und \SIrange{4}{8}{\giga\hertz}.
	Bei beiden ist die Durchlassdämpfung dort nahezu null und die Sperrdämpfung mit mindestens \SI{30}{\decibel} größer als die dafür angegeben \SI{20}{\decibel}\footnote{Genauer waren die "schlechtesten" Werte für die Isolierung bei unterschiedlichen Frequenzbereichen und einer Temperatur von \SI{25}{\celsius} angegeben. Sie alle lagen über \SI{20}{\decibel} und unter \SI{22}{\decibel}.}.
	Das periodische Verhalten in der Sperrdämpfung ist auf Interferenzeffekte zurückzuführen, die durch nicht richtig abgeschlossene Kabel und die folgenden Reflexion entstanden sind.
	
	\
	
	Für beide Kabel konnte eine Ausbreitungsgeschwindigkeit, sowie eine Permittivitätszahl bestimmt werden.
	Konkrete Werte sind in Tab. \ref{tab:kabel} aufgeführt.
	Der große Unterschied von Permittivitätszahlen beider Kabel lässt darauf schließen, dass in den Kabeln unterschiedliche Dielektrika verwendet worden sind.
	Durch die hohe Permittivitätszahl des hellen Kabels, könnte es sich bei dessen Dielektrikum um Teflon handeln.
	Über das Material des dunklen Kabels kann außer der groben Vermutung eines in Schaum eingebrachten Isoliergases keine Aussage gemacht werden (Vergleich mittels \cite{permit1} und \cite{permit2}).
	
	\
	
	Abschließend lässt sich sagen, dass die Ziele dieser Untersuchung erreicht wurden, da die ermittelten Dämpfungen der Bauteile gut mit den Angaben übereinstimmen, teilweise sogar größere Bandbreiten ergeben und auch die Eigenschaften der Koaxialkabel eindeutig bestimmt werden konnten.
	 

	%	}{}
	
	% --- Anhang einbinden
	\IfFileExists{tex/20_Anhang.tex}{
		\newpage
		\subsection{Unsicherheiten}\label{VGuD}

Jegliche Unsicherheiten werden nach GUM\cite{gum} bestimmt und berechnet.
Die Gleichungen dazu finden sich in \cref{fig:GUM_combine} und \cref{fig:GUM_formula}.
Für die Unsicherheitsrechnungen wurde die Python Bibliothek \texttt{uncertainties} herangezogen, welche den Richtlinien des GUM folgt.

Zur Erstellung von Anpassungskurven wird das Python-Paket \texttt{scipy.odr} verwendet, welches unter anderem die Methoden \texttt{scipy.odr.Model()}, \texttt{scipy.odr.RealData()} und \texttt{scipy.odr.ODR()} zur Verfügung stellt.
Dabei wird auf die sogenannte orthogonale lineare Regression (engl. \emph{Orthogonal Distance Regression} (Abkürzung: ODR)) zurückgegriffen, welche auf der Methode der kleinsten Quadrate basiert und einen modifizierten Levenberg-Marquardt-Algorithmus darstellt.
Für die Parameter von Anpassungskurven und deren Unsicherheiten werden die $x$- und $y$-Unsicherheiten der anzunähernden Werte berücksichtigt und entsprechend gewichtet.
Bei digitalen Messungen wird eine Rechteckverteilung mit $\sigma_X = \frac{\delta X}{2\sqrt{3}}$ und bei analogem Ablesen eine Dreieckverteilung mit $\sigma_X = \frac{\delta X}{2\sqrt{6}}$ angenommen.
Die konkreten Werte der jeweiligen Fehlerintervalle $\delta X$ werden in den entsprechenden Abschnitten angemerkt.

Die jeweiligen $\delta X$ sind im konkreten Abschnitt zu finden.
\begin{figure}[ht]
	\begin{equation*}
		x = \sum_{i=1}^{N} x_i
		;\quad
		\sigma_x = \sqrt{\sum_{i = 1}^{N} \sigma_{x_i}^2}
	\end{equation*}
	\caption{Formel für kombinierte Unsicherheiten des selben Typs nach GUM.}
	\label{fig:GUM_combine}
\end{figure}

\begin{figure}[ht]
	\begin{align*}
		f = f(x_1, \dots , x_N)
		;\quad
		\sigma_f = \sqrt{\sum_{i = 1}^{N}\left(\pdv{f}{x_i} \sigma_{x_i}\right) ^2}
	\end{align*}
	\caption{Formel für sich fortpflanzende Unsicherheiten nach GUM.}
	\label{fig:GUM_formula}
\end{figure}

\subsection{Diagramme}

\begin{figure}[ht]
	\centering
	\begin{subfigure}[c]{0.45\textwidth}		
		\centering	
		\includegraphics[width=\textwidth]{dat/debeta_Proton.pdf}
	\end{subfigure}
	\begin{subfigure}[c]{0.45\textwidth}
		\centering
		\includegraphics[width=\textwidth]{dat/debeta_Deuteron.pdf}
	\end{subfigure}
	
	\begin{subfigure}[c]{0.45\textwidth}
		\centering
		\includegraphics[width=\textwidth]{dat/debeta_Triton.pdf}
	\end{subfigure}
	\begin{subfigure}[c]{0.45\textwidth}
		\centering
		\includegraphics[width=\textwidth]{dat/debeta_Helium-3.pdf}
	\end{subfigure}
	
	\begin{subfigure}[c]{0.45\textwidth}
		\centering
		\includegraphics[width=\textwidth]{dat/debeta_Alpha.pdf}
	\end{subfigure}
	\begin{subfigure}[c]{0.45\textwidth}
		\centering
		\includegraphics[width=\textwidth]{dat/debeta_Lithium.pdf}
	\end{subfigure}
	
	\caption{$\Delta E \beta^2$ Verteilungen für vermutete Teilchenarten. Eingezeichnet sind die gemessenen Werte, sowie die theoretisch zu erwartenden Verteilungen. $\alpha$-Teilchen geben die Messergebnisse am besten wieder.}
	\label{fig:debeta_full}
\end{figure}

\subsection{Fit-Parameter}
\label{sec:fitval}

\begin{table}[H]
	\centering
	\caption{Fit-Parameter für die normierte Doppelgaussfunktion \mbox{$n(E) = A_1 \cdot \exp{-\frac{(E - E_1)^2}{2 \sigma_1^2}} + A_2 \cdot \exp{-\frac{(E - E_2)^2}{2 \sigma_2^2}} + A_0$} der Dickebestimmung.}
	\label{tab:fitval1}
	\input{dat/m3_fitdaten.txt}
\end{table}
	}{}
	
	\clearpage
	\printbibliography
	
\end{document}
