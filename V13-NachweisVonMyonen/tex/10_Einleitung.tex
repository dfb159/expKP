\section{Einleitung}

Jede Messung in der Physik hat als Ergebnis niemals einen exakten Wert, sondern es lässt sich immer nur ein Intervall angeben, innerhalb dessen der tatsächliche Wert mit einer bestimmten Wahrscheinlichkeit liegt.\cite{gum} Ziel einer jeden Messung ist es also, das Unsicherheitsintervall so klein wie möglich zu halten. Der im Folgenden untersuchte Mößbauer-Effekt sorgt dafür, dass die Übergangsenergien zwischen verschiedenen Anregungsniveaus eines Atomkerns mit einer Unsicherheit gemessen werden können, die im Bereich der natürlichen Linienbreite liegt. Dies stellt eine extrem hohe Präzision dar. 

Beispielsweise wird bei dem in diesem Versuch relevanten $\gamma$-Zerfall von angeregtem Eisen in den Grundzustand ein Photon mit der Energie $14,4\,$keV emittiert. Die natürliche Linienbreite beträgt $4,7\cdot 10^{-9}\,$eV. Dies bedeutet eine relative Genauigkeit der Messung von der Größenordnung $10^{-12}$.\cite{schatz}

Grundlegende Idee des Mößbauer-Effektes ist es, das Atom, welches das $\gamma$-Quant emittiert bzw. absorbiert, in ein Gitter einzubauen. Dadurch kann verhindert werden, dass sich die Linie der Übergangsenergie aufgrund des Rückstoßes zwischen Photon und Kern verschiebt bzw. verbreitert. Die Wahrscheinlichkeit, dass eine rückstoßfreie Emission bzw. Absorption stattfindet, wird durch den sogenannten Debye-Waller-Faktor beschrieben.

Neben der reinen Beobachtung des Mößbauer-Effektes wird dieser außerdem dazu verwendet, die Isomerieverschiebung und Aufspaltung der Kernenergieniveaus verschiedener Proben zu bestimmen, sowie die Probe selbst zu identifizieren. Daraus kann dann z.\,B. das magnetische Moment der angeregten Zustände berechnet werden.


%Soll enthalten:

%Zusammenhang:
%Ziel:
%Problem:
%Lösungsansatz: