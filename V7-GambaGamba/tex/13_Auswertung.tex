\section{Datenanalyse}

	Dieser Abschnitt umfasst die Auswertung der aufgenommenen Daten.

\subsection{Diodenkennlinie}
	
	In \cref{fig:kennlinie} ist die bei einer Frequenz von $\SI{5}{\giga\hertz}$ gemessenen Kennlinie in Abhängigkeit des Leistungspegels $L$ dargestellt gemeinsam mit einer der Theorie entsprechenden Anpassung $U(L)$.
	\begin{figure}[h]
		\centering
		\includegraphics[width=0.7\textwidth]{img/Kennlinie.pdf}
		\caption{Kennlinie der Diode bei einer Frequenz von $\SI{5}{\giga\hertz}$}
		\label{fig:kennlinie}
	\end{figure}	
	Um bei der Bestimmung der Dämpfungsmaße von der gemessenen Spannung auf die ausgehende Leistung des Messobjektes zurückschließen zu können, bietet sich eine Anpassung der Kennlinie an. 
	Zur Auswahl einer passenden Funktion sind zuerst einige Vorüberlegungen anzustellen. 
	Die Strom-Spannungabhängigkeit einer Schottky-Diode wird durch den Zusammenhang
	\begin{equation}
		I = I_0 \exp(\frac{q U}{n k_B T}) \coloneqq \frac{\SI{1}{\milli\watt}}{B a} \exp(\frac{U}{a})
		\label{eq:schottky}
	\end{equation}
	beschrieben \cite[S.~91]{schottky}. 
	Da die Messung bei konstanter Temperatur $T$ durchgeführt wurde, bietet sich eine Redefinition der Konstanten an mit $a=\frac{n k_B T}{q}$ und $B=\frac{\SI{1}{\milli\watt}}{I_0 a}$. 
	Über die Leistung $P=U I$ erhält man die Abhängigkeit vom Leistungspegel.
	\begin{align}
		10^{\frac{L}{\SI{10}{\decibelmilliwatt}}} \SI{1}{\milli\watt} &= \frac{\SI{1}{\milli\watt}}{B a} U \exp(\frac{U}{a}) \\
		\Rightarrow B\cdot 10^{\frac{L}{\SI{10}{\decibelmilliwatt}}} &=f\left(\frac{U}{a}\right)
	\end{align}
	Dabei entspricht $f\left(\frac{U}{a}\right)$ dem Inversen der Lambert W-Funktion \cite{lambertW}. 
	Dies führt zu
	\begin{equation}
		U_k(L)=a W_k\left(B\cdot 10^{\frac{L}{\SI{10}{\decibelmilliwatt}}}\right),
	\end{equation}
	wobei der Parameter den $k$-ten Ast der Funktion beschreibt. 
	Zur Durchführung der Least-Square Anpassung wurde die Implementation der W-Funktion des \texttt{scipy.special} Pakets für $k=0$ genutzt \cite{scipyLambertW}. 
	Daraus lassen sich die Parameter auf $a=\SI{0,772\pm0,023}{\volt}$ und $B=\SI{0,310\pm0,019}{}$ bestimmen. 

\subsection{Bauteile der Hochfrequenztechnik}
	
	Der anliegende Leistungspegel von $\SI{10}{\decibelmilliwatt}$ entspricht einer eingehenden Leistung von $P_\text{in}=\SI{10}{\milli\watt}$. 
	Mit Hilfe der im vorigen Abschnitt bestimmten Parameter $a$ und $B$ lässt sich nun nach \cref{eq:schottky} auf die an der Diode abfallende Leistung $P_\text{out}$ über
	\begin{equation}
		P_\text{out}=\frac{\SI{1}{\milli\watt}}{a B} U \exp(\frac{U}{a})
	\end{equation}
	zurückschließen. 
	Damit lässt sich im Allgemeinen das Dämpfungsmaß über
	\begin{equation}
		\alpha=10 \log_{10}\left(\frac{P_\text{in}}{P_\text{out}}\right) \si{\decibel}
	\end{equation}
	ausdrücken. 
	In \cref{fig:isolator_daempfung} ist die Sperr- und Durchlassdämpfung für den Isolator dargestellt. 
	Daraus leitet sich per Augenmaß eine Bandbreite $B_I=\SI{4}{\giga\hertz}$ von $\SI{4}{\giga\hertz}$ bis $\SI{8}{\giga\hertz}$ ab.
	\begin{figure}[H]
		\centering
		\includegraphics[width=0.7\textwidth]{img/Isolator_Daempfung.pdf}
		\caption{Sperr- sowie Durchlassdämpfung des Isolators in Abhängigkeit der Frequenz des Signals}
		\label{fig:isolator_daempfung}
	\end{figure}
	Wie in \cref{fig:richtkoppler_daempfung} zu sehen ist, ist der Richtkoppler über den gesamten gemessenen Frequenzbereich einsetzbar.
	\begin{figure}[H]
		\centering
		\includegraphics[width=0.7\textwidth]{img/Richtkoppler_Daempfung.pdf}
		\caption{Koppel-, Einfüge- und Isolationsdämpfung des Richtleiters in Abhängigkeit der Frequenz des Signals}
		\label{fig:richtkoppler_daempfung}
	\end{figure}
	Die beiden Dämpfungsmaße der untersuchten Zirkulatoren sind in \cref{fig:zirkulator_daempfung} dargestellt. 
	Beide Zirkulatoren weisen eine Bandbreite von $B_\text{Z}=\SI{4}{\giga\hertz}$ auf. 
	Allerdings gilt diese für den ersten ab $\SI{1}{\giga\hertz}$, während der zweite mit erst ab $\SI{4}{\giga\hertz}$ einsetzbar ist.
	\begin{figure}[H]
		\centering
		\includegraphics[width=0.7\textwidth]{img/Zirkulator_Daempfung.pdf}
		\caption{Sperr- und Durchlassdämpfung der beiden Zirkulatoren in Abhängigkeit der Frequenz des Signals}
		\label{fig:zirkulator_daempfung}
	\end{figure}

\subsection{Messen der stehenden Wellen/Resonanzen}
	Im folgenden Teil werden zwei Koaxialkabel untersucht, welche im Folgenden als dunkles und helles Kabel bezeichnet werden.
	Dazu wird über einen Koppler der Ausgang des Zirkulators kapazitiv mit dem Kabel verbunden.
	Der Koppler reflektiert einen Großteil des einfallenden Signals und der transmittierte Anteil bildet im Kabel stehende Wellen aus.
	Im Resonanzfall liegen diese genau so, dass das transmittierte Signal immer weiter ansteigt und dem reflektierten Signal destruktiv entgegenwirkt.
	In beiden Messungen wird das jeweilige Kabel am Ende kurzgeschlossenen.
	
	\begin{figure}[H]
		\centering
		\begin{subfigure}[c]{\textwidth}		
			\centering	
			\includegraphics[width=0.8\textwidth]{data/kopplerDunkel.pdf}
			\subcaption{}		
		\end{subfigure}
		
		\begin{subfigure}[c]{\textwidth}
			\centering
			\includegraphics[width=0.8\textwidth]{data/kopplerHell.pdf}
			\subcaption{}
		\end{subfigure}
		
		\caption{Messung von zwei Koaxialkabeln. Oben sind die Daten des dunklen Kabels, unten die des hellen Kabels dargestellt. Des Weiteren sind jeweils die Frequenzen von zwei Resonanzen markiert.}
		\label{fig:resonanz}
	\end{figure}

	Bei dem Vergleich der beiden Kabel fällt schnell auf, dass das dunkle Kabel mehr Resonanzen im beobachteten Frequenzintervall aufweist.
	Quantitativ ist die Frequenzdifferenz zwischen zwei Resonanzen durch den Mittelwert $\Delta f = \frac{f_n - f_1}{n-1}$ gegeben.
	Beim dunklen Kabel beträgt diese $\Delta f_\text{D} = \input{data/SIdunkelFrequenz.txt}$, während sie beim hellen Kabel $\Delta f_\text{H} = \input{data/SIhellFrequenz.txt}$ beträgt.
	Die Positionen der Resonanzfrequenzen wurden hier manuell in die Mitte eines Spannungseinbruchs gelegt.
	Die Unsicherheit einer Frequenz wurde durch die Breite eines Einbruchs mit $\SI{0,01}{\giga\hertz}$ abgeschätzt.
	
	Aus den ermittelten Frequenzabständen der Resonanzen kann nun nach $c = 2 l \Delta f$, Gl. (3.5) \cite{wwu} die Ausbreitungsgeschwindigkeit der Mikrowellen im Koaxialkabel berechnet werden.
	Die Länge eines Kabels $l$ ist mit einem Maßband gemessen worden und die Unsicherheit wird mit $\Delta l = \SI{1}{\centi\meter}$ abgeschätzt, da die Reflektionspositionen im Koppler und im Abschlusswiderstand nicht genau bestimmt werden können.
	Da die Permeabilitätszahl des nicht ferromagnetischen Dielektrikums $\mu_\text{r} \approx 1$ ist, kann aus der Ausbreitungsgeschwindigkeit jetzt die Permittivitätszahl des Dielektrikums $\epsilon_\text{r} = \left( \frac{c_0}{c} \right) ^2$, Gl. (3.7) \cite{wwu} bestimmt werden.
	Die Werte für beide Kabel sind in Tab. \ref{tab:kabel} zusammengefasst.
	
	\begin{table}[H]
		\centering
		\caption{Gemessene und daraus ermittelte Eigenschaften der beiden Koaxialkabel.} 
		\label{tab:kabel}
		\begin{tabular}{l|c|c}
			                                       &           dunkles Kabel           &          helles Kabel           \\ \hline
			Kabellänge $l$                         &   \SI{122,5 +- 1}{\centi\meter}   &  \SI{101,5 +- 1}{\centi\meter}  \\
			Resonanzabstände $\Delta f$            & \input{data/SIdunkelFrequenz.txt} & \input{data/SIhellFrequenz.txt} \\
			Ausbreitungsgeschwindigkeit $c$        &    \input{data/SIdunkelC.txt}     &    \input{data/SIhellC.txt}     \\
			Permittivitätszahl $\epsilon_\text{r}$ &     \input{data/dunkelE.txt}      &     \input{data/hellE.txt}
		\end{tabular}
	\end{table}
	