% Autor: Simon May
% Datum: 2017-10-04

% --- Pakete einbinden
% --- Pakete erweitern LaTeX um zusätzliche Funktionen.
%     Dies ist ein Satz nützlicher Pakete.

% Silbentrennung etc.; Sprache wird durch Option bei \documentclass festgelegt
\usepackage{babel}
\usepackage{iftex}
\ifLuaTeX
	% Schriftart (Latin Modern)
	\usepackage{fontspec}
	\fontspec{Latin Modern Roman}
\else
	% Verwendung der Zeichentabelle T1 (für Sonderzeichen etc.)
	\usepackage[T1]{fontenc}
	% Legt die Eingabe-Zeichenkodierung fest, z.B. UTF-8
	\usepackage[utf8]{inputenc}
	% Schriftart (Latin Modern)
	\usepackage{lmodern}
	% Zusätzliche Sonderzeichen
	\usepackage{textcomp}
\fi

\usepackage{upgreek}

% Nutzen von +, -, *, / in \setlength u.ä. (z.B. \setlength{\a + 3cm})
\usepackage{calc}
% Wird benötigt, um \ifthenelse zu benutzen
\usepackage{xifthen}
% Optionen für eigene definierte Befehle
\usepackage{xparse}

% Verbessertes Aussehen des Schriftbilds durch kleine Anpassungen
\usepackage{microtype}
% Automatische Formatierung von Daten
\usepackage[useregional]{datetime2}
% Wird für Kopf- und Fußzeile benötigt
\usepackage{scrlayer-scrpage}
% Einfaches Wechseln zwischen unterschiedlichen Zeilenabständen
\usepackage{setspace}
% Optionen für Listen (enumerate, itemize, …)
\usepackage{enumitem}
% Automatische Anführungszeichen
\usepackage{csquotes}
% Zusätzliche Optionen für Tabellen (tabular)
\usepackage{array}

% Mathepaket (intlimits: Grenzen über/unter Integralzeichen)
\usepackage[intlimits]{amsmath}
% Mathe-Symbole, \mathbb etc.
\usepackage{amssymb}
% Weitere Mathebefehle
\usepackage{mathtools}
% „Schöne“ Brüche im Fließtext
\usepackage{xfrac}
% Ermöglicht die Nutzung von \SI{Zahl}{Einheit} u.a.
\usepackage{siunitx}
% Ermöglicht Nutzung von \pdv als Ableitungen
\usepackage{physics}
% Definition von Unicode-Symbolen; Nach [utf8]inputenc laden!
\usepackage{newunicodechar}
% Unicode-Formeln mit pdfLaTeX
\input{tex/99_pdflatex_unicode-math.tex}

% Farben
\usepackage{xcolor}
% Einbinden von Grafiken (\includegraphics)
\usepackage{graphicx}
% .tex-Dateien mit \includegraphics einbinden
\usepackage{gincltex}
% Größere Freiheiten bei Dateinamen mit \includegraphics
\usepackage{grffile}
% Abbildungen im Fließtext
\usepackage{wrapfig}
% Zitieren, Bibliographie (Biber als Bibliographie-Programm verwenden!)
\usepackage[backend=biber]{biblatex}
% Abbildungen nebeneinander (subfigure, subtable)
\usepackage{subcaption}
\usepackage{float}
\usepackage{booktabs}

% Verlinkt Textstellen im PDF-Dokument (sollte am Ende geladen werden)
\usepackage[unicode]{hyperref}
% „Schlaue“ Referenzen (nach hyperref laden!)
\usepackage{cleveref}
%PDF einbinden
%\usepackage{pdfpages}
%Graphiken zeichnen
%\usepackage{tikz}
%\usetikzlibrary{angles,quotes,babel,3d}
