% Autor: Simon May
% Datum: 2016-10-13
% Der Befehl \newcommand kann auch benutzt werden, um „Variablen“ zu definieren:

% Nummer laut Praktikumsheft:
\newcommand*{\varNum}{V7}
% Name laut Praktikumsheft:
\newcommand*{\varName}{$\gamma$-$\gamma$-Winkelkorrelation}
% Datum der Durchführung:
\newcommand*{\varDatum}{18.11.2019}
% Autoren des Protokolls:
\newcommand*{\varAutor}{C. Lippe, J. Sigrist, J. T. Zarnitz}
\newcommand*{\varNameA}{Chris Lippe}
\newcommand*{\varNameB}{Jonathan Sigrist}
\newcommand*{\varNameC}{Jannik Tim Zarnitz}
% Nummer der eigenen Gruppe:
\newcommand*{\varGruppe}{Gruppe Ma-A-06}
% E-Mail-Adressen der Autoren (kommagetrennt ohne Leerzeichen!):
\newcommand{\varEmail}{c\_lipp02@wwu.de,j\_sigrist@wwu.de,j\_zarn02@wwu.de}
\newcommand{\varEmailA}{c\_lipp02@wwu.de}
\newcommand{\varEmailB}{j\_sigrist@wwu.de}
\newcommand{\varEmailC}{j\_zarn02@wwu.de}
%betreuer Name
\newcommand{\varBetreuer}{\normalsize betreut von Benjamin Hetz}
% E-Mail-Adresse anzeigen (true/false):
\newcommand*{\varZeigeEmail}{true}
% Kopfzeile anzeigen (true/false):
\newcommand*{\varZeigeKopfzeile}{true}
% Inhaltsverzeichnis anzeigen (true/false):
\newcommand*{\varZeigeInhaltsverzeichnis}{true}
% Literaturverzeichnis anzeigen (true/false):
\newcommand*{\varZeigeLiteraturverzeichnis}{true}

\newboolean{showEmail}
\setboolean{showEmail}{\varZeigeEmail}
\newboolean{showHeader}
\setboolean{showHeader}{\varZeigeKopfzeile}
\newboolean{showTOC}
\setboolean{showTOC}{\varZeigeInhaltsverzeichnis}
\newboolean{showBibliography}
\setboolean{showBibliography}{\varZeigeLiteraturverzeichnis}
