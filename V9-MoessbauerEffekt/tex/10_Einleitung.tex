\section{Einleitung}

Jede Messung in der Physik hat als Ergebnis niemals einen exakten Wert, sondern es lässt sich immer nur ein Intervall angeben, innerhalb dessen der tatsächliche Wert mit einer bestimmten Wahrscheinlichkeit liegt.$^\text{\cite{gum}}$ Ziel einer jeden Messung ist es also, das Unsicherheitsintervall so klein wie möglich zu halten. Der im Folgenden untersuchte Mößbauer-Effekt sorgt dafür, dass die Übergangsenergien zwischen verschiedenen Anregungsniveaus mit einer Unsicherheit gemessen werden können, die im Bereich der natürlichen Linienbreite liegt. Dies stellt eine extrem hohe Präzision dar. 

